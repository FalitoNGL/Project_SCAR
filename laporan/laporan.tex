\documentclass[12pt,a4paper]{article}

% ============================================================
% PACKAGES
% ============================================================
\usepackage[indonesian]{babel}
\usepackage[utf8]{inputenc}
\usepackage[T1]{fontenc}
\usepackage{times}
\usepackage[margin=2.5cm]{geometry}
\usepackage{graphicx}
\usepackage{float}
\usepackage{booktabs}
\usepackage{tabularx}
\usepackage{longtable}
\usepackage{listings}
\usepackage{xcolor}
\usepackage{hyperref}
\usepackage{amsmath}
\usepackage{setspace}
\usepackage{titlesec}
\usepackage{fancyhdr}
\usepackage{caption}
\usepackage{enumitem}
\usepackage{tikz}
\usepackage{indentfirst}
\usetikzlibrary{shapes.geometric, arrows, positioning, fit}

% ============================================================
% KONFIGURASI
% ============================================================
\setstretch{1.05}
\setlength{\parindent}{1.25cm}
\setlength{\parskip}{0.5em}

\renewcommand{\thesection}{BAB \Roman{section}}
\renewcommand{\thesubsection}{\arabic{section}.\arabic{subsection}}
\renewcommand{\thesubsubsection}{\arabic{section}.\arabic{subsection}.\arabic{subsubsection}}

\titleformat{\section}[block]
  {\normalfont\Large\bfseries\centering}
  {BAB \Roman{section}\\}{0pt}
  {\vspace{0.3cm}\MakeUppercase}
\titlespacing*{\section}{0pt}{1cm}{0.8cm}

% Fix TOC width for "BAB I", "BAB II", etc.
\makeatletter
\renewcommand{\l@section}{\@dottedtocline{1}{0em}{5em}}
\renewcommand{\l@subsection}{\@dottedtocline{2}{5em}{2.5em}}
\renewcommand{\l@subsubsection}{\@dottedtocline{3}{7.5em}{3.2em}}
\makeatother

\hypersetup{
    colorlinks=true,
    linkcolor=black,
    filecolor=black,
    urlcolor=black,
    citecolor=black
}

\lstdefinestyle{pythonstyle}{
    language=Python,
    backgroundcolor=\color{gray!10},
    basicstyle=\ttfamily\small,
    keywordstyle=\color{blue}\bfseries,
    stringstyle=\color{red},
    commentstyle=\color{green!60!black}\itshape,
    numberstyle=\tiny\color{gray},
    numbers=left,
    numbersep=8pt,
    frame=single,
    breaklines=true,
    showstringspaces=false,
    tabsize=4,
    captionpos=b
}
\lstset{style=pythonstyle}

\pagestyle{fancy}
\fancyhf{}
\fancyfoot[C]{\thepage}
\renewcommand{\headrulewidth}{0pt}

\begin{document}

% ============================================================
% HALAMAN JUDUL
% ============================================================
\begin{titlepage}
    \centering
    {\LARGE\textbf{LAPORAN PROYEK AKHIR}\par}
    \vspace{0.3cm}
    {\Large\textbf{PEMROGRAMAN JARINGAN}\par}
    \vspace{0.8cm}
    {\Huge\textbf{S.C.A.R.}\par}
    \vspace{0.3cm}
    {\Large\textit{System for Countering Automated Reconnaissance}\par}
    \vspace{0.5cm}
    {\large Active Defense Honeypot dengan Multi-Layer AI\\dan Tarpit Resource Exhaustion\par}
    \vspace{0.5cm}
    \begin{figure}[H]
        \centering
        \includegraphics[width=5cm]{logo.png}
    \end{figure}
    \vspace{0.3cm}

    {\large\textbf{Disusun oleh:}\par}
    \vspace{0.5cm}
    {\large
    \begin{tabular}{ll}
        Falito Eriano Nainggolan & (2423102023) \\
        Luklu Miranda & (2423102039) \\
        Reiza Gerrard Rizki Ramadhan & (2423102070) \\
    \end{tabular}
    \par}

    \vspace{0.5cm}

    {\large\textbf{Dosen Pengampu:}\par}
    \vspace{0.3cm}
    {\large Bapak Agus Winarno, S.S.T.TP, M.T.\par}

    \vspace{0.8cm}

    {\large
    \textbf{Tingkat II Rekayasa Keamanan Siber A}\\
    \vspace{0.3cm}
    \textbf{POLITEKNIK SIBER DAN SANDI NEGARA}\\
    \vspace{0.3cm}
    \textbf{Semester Gasal 2025/2026}\\
    }
\end{titlepage}

\newpage
\tableofcontents
\newpage
\listoffigures
\listoftables
\newpage

% ============================================================
% BAB 1: PENDAHULUAN
% ============================================================
\clearpage
\section{Pendahuluan}

\subsection{Latar Belakang}

Dalam konteks pemrograman jaringan, server HTTP merupakan salah satu komponen fundamental yang berperan sebagai titik masuk utama komunikasi antara klien dan layanan. Server menerima \textit{request} dari klien melalui protokol TCP/IP, memproses \textit{request} tersebut, dan mengirimkan \textit{response} melalui \textit{socket} jaringan. Akan tetapi, setiap server yang terekspos ke jaringan publik secara inheren menjadi vektor serangan potensial bagi aktor ancaman siber.

Berdasarkan data yang dipublikasikan dalam \textit{Verizon Data Breach Investigations Report} (2024), serangan terhadap aplikasi web mengalami peningkatan sebesar 30\% secara tahunan. Server yang terhubung ke internet menerima ratusan \textit{request} pemindaian otomatis setiap harinya, yang berasal dari \textit{bot}, \textit{scanner}, maupun \textit{script} berbahaya. Serangan seperti \textit{SQL Injection}, \textit{Directory Traversal}, dan \textit{Automated Reconnaissance} merupakan ancaman utama yang harus dimitigasi oleh setiap pengelola server jaringan.

Konsep \textit{honeypot}, yang diperkenalkan pertama kali oleh Lance Spitzner dalam karyanya \textit{``Honeypots: Tracking Hackers''} (2003), merujuk pada server umpan yang sengaja dirancang untuk menarik dan mempelajari perilaku penyerang. Konsep ini memanfaatkan pemahaman mendalam tentang protokol jaringan --- meliputi mekanisme \textit{TCP three-way handshake}, siklus \textit{HTTP request-response}, serta manipulasi pada level protokol untuk tujuan defensif.

Proyek \textbf{S.C.A.R. (\textit{System for Countering Automated Reconnaissance})} mengembangkan konsep \textit{honeypot} konvensional dengan mengimplementasikan paradigma \textbf{\textit{Active Defense}}. Berbeda dengan \textit{honeypot} tradisional yang bersifat pasif (hanya mencatat aktivitas serangan), S.C.A.R. secara aktif memanfaatkan karakteristik protokol HTTP dan mekanisme koneksi TCP untuk \textbf{menjebak} penyerang melalui teknik \textit{Tarpit} (\textit{Resource Exhaustion}). Teknik ini mengeksploitasi spesifikasi \textit{chunked transfer encoding} pada HTTP/1.1 untuk mempertahankan koneksi TCP tetap terbuka dan menguras sumber daya komputasi di sisi penyerang --- sebuah pendekatan yang dikenal sebagai \textit{TCP state exhaustion} dan \textit{HTTP response stream manipulation}.

Secara spesifik, proyek ini mengintegrasikan aspek-aspek inti pemrograman jaringan sebagai berikut:
\begin{itemize}
    \item \textbf{\textit{Socket Programming}} --- Pembangunan server HTTP menggunakan modul \texttt{http.server} dan \texttt{socketserver} Python yang beroperasi pada level \textit{socket} TCP.
    \item \textbf{Manipulasi Protokol HTTP} --- Pemanfaatan \textit{headers}, \textit{status codes}, dan \textit{transfer encoding} untuk keperluan pertahanan aktif.
    \item \textbf{Integrasi REST API} --- Komunikasi dengan layanan eksternal (AbuseIPDB, Telegram Bot) melalui protokol HTTP sebagai \textit{client}.
    \item \textbf{\textit{Concurrency}} --- Penerapan \textit{multi-threading} untuk penanganan koneksi simultan dan operasi I/O \textit{non-blocking}.
    \item \textbf{\textit{Machine Learning}} --- Pemanfaatan model AI sebagai komponen analisis lalu lintas jaringan secara \textit{real-time}.
\end{itemize}

\subsection{Rumusan Masalah}

\begin{enumerate}[label=\arabic*.]
    \item Bagaimana membangun server HTTP menggunakan Python yang mampu menangani request masuk, menganalisis kontennya, dan memberikan respons yang sesuai secara \textit{real-time}?
    \item Bagaimana memanipulasi protokol HTTP (khususnya \textit{chunked transfer encoding} dan \textit{custom headers}) untuk mengimplementasikan mekanisme \textit{Tarpit} yang efektif menghabiskan sumber daya penyerang?
    \item Bagaimana mengintegrasikan layanan API eksternal (AbuseIPDB, Telegram Bot API) ke dalam server melalui komunikasi HTTP \textit{client-side}?
    \item Bagaimana menerapkan \textit{multi-threading} agar server tetap responsif saat menangani koneksi tarpit dan pengiriman notifikasi secara bersamaan?
\end{enumerate}

\subsection{Tujuan Proyek}

\begin{enumerate}[label=\arabic*.]
    \item Membangun server HTTP aktif berbasis Python yang berfungsi sebagai honeypot dengan kemampuan \textit{Active Defense}.
    \item Mengimplementasikan mekanisme \textit{Tarpit} yang memanfaatkan \textit{chunked transfer encoding} HTTP untuk menjebak dan menghabiskan sumber daya penyerang.
    \item Mengintegrasikan AI (\textit{Machine Learning}) sebagai komponen analisis lalu lintas jaringan untuk mendeteksi serangan secara otomatis.
    \item Mengintegrasikan API eksternal (AbuseIPDB dan Telegram Bot) melalui HTTP REST API untuk \textit{threat intelligence} dan notifikasi.
    \item Menerapkan arsitektur \textit{multi-threaded} untuk menangani operasi I/O non-blocking.
\end{enumerate}

\subsection{Batasan Proyek}

\begin{enumerate}[label=\arabic*.]
    \item Server dibangun menggunakan modul \texttt{http.server} dari pustaka standar Python.
    \item Sistem dirancang untuk berjalan pada lingkungan lokal atau server tunggal.
    \item Pengujian dilakukan menggunakan \textit{script} simulasi serangan.
    \item Model AI dilatih menggunakan dataset publik yang tersedia secara gratis.
\end{enumerate}

% ============================================================
% BAB 2: TINJAUAN PUSTAKA
% ============================================================
\clearpage
\section{Tinjauan Pustaka}

\subsection{Protokol HTTP dan Arsitektur Request-Response}

HTTP (\textit{Hypertext Transfer Protocol}), didefinisikan dalam RFC 2616, adalah protokol \textit{application layer} yang berjalan di atas koneksi TCP. Setiap komunikasi HTTP mengikuti pola \textit{request-response}:

\begin{enumerate}[label=\arabic*.]
    \item \textbf{Klien} membuka koneksi TCP ke server (biasanya port 80 atau 8080).
    \item Klien mengirim \textbf{HTTP Request} yang terdiri dari: \textit{request line} (metode, path, versi), \textit{headers}, dan opsional \textit{body}.
    \item \textbf{Server} menerima request, memproses, dan mengirim \textbf{HTTP Response} yang terdiri dari: \textit{status line} (kode status), \textit{headers}, dan \textit{body}.
    \item Koneksi bisa dipertahankan (\textit{keep-alive}) atau ditutup.
\end{enumerate}

Pemahaman mendalam tentang siklus ini sangat penting dalam S.C.A.R. karena mekanisme tarpit bekerja dengan cara memanipulasi tahap ke-3 --- mengirim response yang \textbf{tidak pernah selesai}.

\subsection{Transfer-Encoding: Chunked}

\textit{Chunked Transfer Encoding} (RFC 2616, Section 3.6.1) adalah mekanisme HTTP yang memungkinkan server mengirim data secara bertahap tanpa harus mengetahui total ukuran konten di awal. Format pengiriman:

\begin{lstlisting}[language={}, caption={Format Chunked Transfer Encoding}]
HTTP/1.1 200 OK
Transfer-Encoding: chunked

[ukuran chunk dalam hex]\r\n
[data chunk]\r\n
[ukuran chunk berikutnya]\r\n
[data chunk berikutnya]\r\n
0\r\n                          (chunk terakhir, ukuran 0)
\r\n                           (trailer kosong)
\end{lstlisting}

Fitur kunci yang dieksploitasi oleh S.C.A.R.: selama server belum mengirim chunk terakhir (ukuran 0), \textbf{klien wajib menunggu} data tambahan. Ini adalah dasar mekanisme tarpit --- server mengirim data sampah secara terus-menerus dengan jeda yang panjang, memaksa klien tetap terhubung.

\subsection{Socket Programming dan http.server Python}

Modul \texttt{http.server} Python dibangun di atas modul \texttt{socketserver} yang menyediakan abstraksi \textit{socket} TCP. Arsitekturnya:

\begin{itemize}
    \item \texttt{TCPServer}: Mengelola \textit{socket} server, melakukan \texttt{bind()} dan \texttt{listen()} pada port tertentu.
    \item \texttt{BaseHTTPRequestHandler}: Menangani setiap koneksi TCP yang masuk, mem-\textit{parse} HTTP request, dan memanggil metode \texttt{do\_GET()}, \texttt{do\_POST()}, dll.
    \item \texttt{self.wfile}: \textit{File-like object} yang terhubung langsung ke \textit{socket} klien, memungkinkan penulisan data langsung ke koneksi TCP.
\end{itemize}

Akses langsung ke \texttt{self.wfile} sangat krusial untuk implementasi tarpit, karena memungkinkan pengiriman data byte-per-byte langsung ke \textit{socket} tanpa buffering HTTP standar.

\subsection{Honeypot dalam Keamanan Jaringan}

\textit{Honeypot}, menurut Lance Spitzner, adalah ``sumber daya keamanan informasi yang nilainya terletak pada interaksi yang tidak sah.'' Dari perspektif pemrograman jaringan, honeypot adalah server yang sengaja menyediakan \textit{service} palsu (HTTP, SSH, FTP) untuk menarik dan mempelajari penyerang.

Klasifikasi berdasarkan interaksi:
\begin{itemize}
    \item \textbf{Low-Interaction}: Mengemulasi layanan jaringan secara terbatas. Aman, mudah di-\textit{deploy}.
    \item \textbf{High-Interaction}: Menyediakan layanan nyata. Informatif tetapi berisiko.
\end{itemize}

S.C.A.R. termasuk \textit{Low-Interaction Honeypot} dengan kemampuan \textit{Active Defense} --- memadukan keamanan \textit{low-interaction} dengan agresivitas \textit{active defense}.

\subsection{Tarpit sebagai Mekanisme Pertahanan Aktif}

\textit{Tarpit} (atau \textit{tar pit}) merupakan teknik manipulasi koneksi jaringan yang bertujuan memperlambat komunikasi secara drastis melalui \textit{TCP state exhaustion}. Terminologi ini terinspirasi dari fenomena lubang aspal alami (\textit{La Brea Tar Pits}) yang menjebak organisme purba. Tom Liston mengimplementasikan konsep ini pertama kali dalam proyek \textit{LaBrea Tarpit} (2003) sebagai mekanisme pertahanan terhadap propagasi \textit{worm} jaringan.

Dari perspektif pemrograman jaringan, mekanisme tarpit beroperasi melalui tiga prinsip utama:
\begin{enumerate}[label=\arabic*.]
    \item \textbf{\textit{TCP State Exhaustion}} --- Server mempertahankan koneksi TCP dalam state \texttt{ESTABLISHED} tanpa batas waktu, sehingga \textit{socket file descriptor}, \textit{thread}, dan alokasi memori di sisi klien tetap terpakai dan tidak dapat didaur ulang.
    \item \textbf{\textit{HTTP Response Stream Manipulation}} --- Server mengeksploitasi mekanisme \textit{chunked transfer encoding} untuk mengirimkan data \textit{garbage} secara kontinu dengan interval yang diperpanjang (\textit{slow drip}), memaksa klien tetap dalam mode \textit{receive-wait}.
    \item \textbf{\textit{Connection Pool Saturation}} --- Setiap koneksi tarpit mengikat satu \textit{connection pool entry} pada sisi penyerang, sehingga kapasitas \textit{concurrent connection} penyerang terdegradasi secara progresif.
\end{enumerate}

Keunggulan pendekatan tarpit dibandingkan pemblokiran konvensional (\textit{firewall drop/reject}):
\begin{itemize}
    \item Penyerang \textbf{tidak mengetahui} bahwa identitasnya telah terdeteksi, karena server tetap mengirimkan respons awal \texttt{HTTP 200 OK} yang valid.
    \item \textit{Tool} otomatis (SQLMap, Nikto, DirBuster) mengalami kondisi \textit{hang} akibat menunggu penyelesaian transfer yang tidak pernah berakhir.
    \item Kecepatan \textit{scanning} otomatis terdegradasi secara signifikan, memberikan waktu tambahan bagi administrator untuk melakukan mitigasi.
\end{itemize}

\subsection{REST API dan Integrasi Layanan Eksternal}

REST (\textit{Representational State Transfer}) API adalah arsitektur komunikasi jaringan yang menggunakan protokol HTTP standar untuk pertukaran data antar sistem. Dalam S.C.A.R., server bertindak sebagai \textbf{HTTP client} ketika berkomunikasi dengan layanan eksternal:

\begin{itemize}
    \item \textbf{AbuseIPDB API}: Server mengirim HTTP GET request ke \texttt{api.abuseipdb.com} untuk memeriksa reputasi IP. Response berupa JSON yang berisi skor kepercayaan.
    \item \textbf{Telegram Bot API}: Server mengirim HTTP POST request ke \texttt{api.telegram.org} untuk mengirim notifikasi. Data dikirim dalam format JSON menggunakan \texttt{Content-Type: application/json}.
\end{itemize}

Kedua API ini mendemonstrasikan konsep penting dalam pemrograman jaringan: sebuah aplikasi dapat bertindak sebagai \textbf{server dan client secara bersamaan} --- menerima koneksi dari penyerang (server) sambil mengirim request ke layanan luar (client).

\subsection{Threading dan Concurrency dalam Server Jaringan}

Dalam pemrograman jaringan, \textit{concurrency} sangat penting karena server harus menangani banyak koneksi secara bersamaan. Python menyediakan modul \texttt{threading} yang memungkinkan eksekusi paralel.

Dalam S.C.A.R., threading digunakan untuk:
\begin{itemize}
    \item \textbf{ThreadingTCPServer}: Setiap koneksi klien ditangani oleh \textit{thread} terpisah, sehingga satu koneksi tarpit tidak memblokir klien lain.
    \item \textbf{Non-blocking API calls}: Pengiriman alert Telegram dilakukan di \textit{background thread} agar proses tarpit tidak tertunda oleh latensi jaringan.
    \item \textbf{Thread-safe caching}: Hasil pengecekan AbuseIPDB di-\textit{cache} menggunakan \texttt{threading.Lock()} untuk menghindari \textit{race condition}.
\end{itemize}

\subsection{\textit{Machine Learning} sebagai Komponen Analisis Lalu Lintas Jaringan}

Dalam konteks pemrograman jaringan, \textit{Machine Learning} (ML) digunakan sebagai komponen analisis cerdas yang memproses data dari lalu lintas jaringan (\textit{HTTP request}) untuk mengklasifikasikan setiap \textit{request} sebagai normal atau ancaman secara otomatis. S.C.A.R. mengintegrasikan empat model ML yang masing-masing menganalisis aspek berbeda dari \textit{request} HTTP:

\begin{table}[H]
\centering
\caption{Model AI dan Data Jaringan yang Dianalisis}
\begin{tabularx}{\textwidth}{llX}
\toprule
\textbf{Model} & \textbf{Algoritma} & \textbf{Data Jaringan yang Dianalisis} \\
\midrule
URL Threat & Logistic Regression & URL \textit{path} dari \textit{request line} HTTP \\
SQL Injection & Logistic Regression & \textit{Query string} dan \textit{body} dari HTTP \textit{request} \\
HTTP Behavior & Random Forest & Fitur statistik: panjang URL, jumlah parameter, metode HTTP \\
Anomaly Detection & Isolation Forest & Profil keseluruhan \textit{request} dibandingkan \textit{baseline} normal \\
\bottomrule
\end{tabularx}
\end{table}

Pemilihan algoritma \textit{Random Forest} (Breiman, 2001) untuk model HTTP Behavior didasarkan pada kemampuannya menangani fitur heterogen (numerik dan kategorikal) serta resistensinya terhadap \textit{overfitting} melalui mekanisme \textit{ensemble} dari multiple \textit{decision trees}. Sementara itu, \textit{Isolation Forest} (Liu et al., 2008) dipilih untuk deteksi anomali karena sifatnya yang \textit{unsupervised} --- tidak memerlukan label serangan eksplisit, melainkan mengidentifikasi \textit{request} yang menyimpang secara statistik dari distribusi normal. Pendekatan ini memungkinkan deteksi pola serangan \textit{zero-day} yang belum terdokumentasi dalam basis data aturan statis.

Pemilihan seluruh algoritma juga didasarkan pada efisiensi komputasi prediksi \textit{real-time}, yang merupakan persyaratan kritis untuk server jaringan --- setiap \textit{request} harus dianalisis tanpa menimbulkan latensi yang terdeteksi oleh klien.

% ============================================================
% BAB 3: PERANCANGAN SISTEM
% ============================================================
\clearpage
\section{Perancangan Sistem}

\subsection{Arsitektur Server}

S.C.A.R. dibangun di atas arsitektur \textit{multi-threaded HTTP server} menggunakan \texttt{ThreadingTCPServer} dari Python. Arsitektur ini memungkinkan server menangani banyak koneksi secara bersamaan --- penting ketika satu koneksi sedang di-tarpit (bisa berlangsung puluhan detik) sementara koneksi lain tetap harus dilayani.

\begin{figure}[H]
\centering
\begin{tikzpicture}[
    node distance=1.2cm,
    layer/.style={draw, rounded corners, minimum width=12cm, minimum height=1cm, align=center, fill=blue!10, font=\small},
    decision/.style={draw, diamond, aspect=2.5, fill=yellow!20, font=\small, align=center},
    action/.style={draw, rounded corners, minimum width=4cm, minimum height=0.8cm, align=center, fill=red!15, font=\small},
    clean/.style={draw, rounded corners, minimum width=4cm, minimum height=0.8cm, align=center, fill=green!15, font=\small},
    arrow/.style={->, thick}
]
    \node[layer] (req) {Incoming TCP Connection (Socket)};
    \node[layer, below=of req] (parse) {HTTP Request Parsing (Method, Path, Headers, Body)};
    \node[layer, below=of parse] (l0) {Layer 0: IP Reputation (AbuseIPDB REST API)};
    \node[layer, below=of l0] (l1) {Layer 1: Reconnaissance Blacklist (Pattern Matching)};
    \node[layer, below=of l1] (l2) {Layer 2--5: AI Threat Fusion Engine};
    \node[decision, below=1.5cm of l2] (dec) {Threat?};
    \node[action, below right=1cm and 0.5cm of dec] (tarpit) {Tarpit via Chunked Encoding};
    \node[clean, below left=1cm and 0.5cm of dec] (serve) {Serve Honeypot HTML};

    \draw[arrow] (req) -- (l0);
    \draw[arrow] (req) -- (parse);
    \draw[arrow] (parse) -- (l0);
    \draw[arrow] (l0) -- (l1);
    \draw[arrow] (l1) -- (l2);
    \draw[arrow] (l2) -- (dec);
    \draw[arrow] (dec) -- node[right, font=\small] {Ya} (tarpit);
    \draw[arrow] (dec) -- node[left, font=\small] {Tidak} (serve);
\end{tikzpicture}
\caption{Alur Pemrosesan Request pada Server S.C.A.R.}
\label{fig:arsitektur}
\end{figure}

\subsection{Alur Komunikasi Jaringan}

Berikut adalah alur komunikasi jaringan lengkap ketika sebuah request masuk ke S.C.A.R.:

\begin{enumerate}[label=\arabic*.]
    \item \textbf{TCP Handshake}: Klien melakukan \textit{three-way handshake} (SYN, SYN-ACK, ACK) dengan server pada port 8000.
    \item \textbf{HTTP Request}: Klien mengirim request melalui \textit{socket} yang telah terhubung. Server mem-\textit{parse} \textit{request line}, \textit{headers}, dan \textit{body}.
    \item \textbf{API Call (Client-Side)}: Server mengirim HTTP GET request ke AbuseIPDB untuk cek reputasi IP (server bertindak sebagai HTTP \textit{client}).
    \item \textbf{Analisis}: Request dianalisis oleh Reconnaissance Blacklist dan 4 model AI.
    \item \textbf{Response}:
    \begin{itemize}
        \item \textit{Clean}: Server mengirim \texttt{HTTP 200 OK} dengan halaman HTML honeypot.
        \item \textit{Threat}: Server mengirim \texttt{HTTP 200 OK} diikuti \textit{chunked encoding} dengan data sampah (tarpit).
    \end{itemize}
    \item \textbf{Telegram Alert}: Jika ancaman terdeteksi, \textit{background thread} mengirim HTTP POST ke Telegram Bot API.
\end{enumerate}

\subsection{Mekanisme Tarpit (\textit{HTTP Response Stream Manipulation})}

Mekanisme tarpit merupakan inti dari strategi pertahanan aktif S.C.A.R. Secara teknis, tarpit beroperasi dengan memanipulasi \textit{HTTP response stream} pada level \textit{socket} TCP:

\begin{lstlisting}[language={}, caption={Urutan Byte yang Dikirim melalui Socket saat Tarpit}]
>> HTTP/1.1 200 OK\r\n              (status line - penyerang pikir berhasil)
>> Transfer-Encoding: chunked\r\n   (memberitahu klien: data akan dikirim bertahap)
>> Server: SCAR-Active-Defense\r\n  (custom header)
>> \r\n                             (akhir header, mulai body)
>> X-Trap-482917: a3f8c1d...\r\n    (data sampah #1)
   [jeda 5 detik]
>> X-Trap-193847: 7bc2e5f...\r\n    (data sampah #2)
   [jeda 5 detik]
>> X-Trap-572910: 1de9f3a...\r\n    (data sampah #3)
   [jeda 5 detik]
   ... (berlanjut hingga 1.000.000 header atau klien disconnect)
\end{lstlisting}

Efektivitas mekanisme ini terletak pada eksploitasi spesifikasi HTTP/1.1 yang mewajibkan klien menunggu penyelesaian transfer:
\begin{itemize}
    \item Klien HTTP \textbf{wajib menunggu} pengiriman \textit{terminating chunk} (ukuran 0) sebelum dapat memproses respons secara utuh, sesuai RFC 2616 Section 3.6.1.
    \item Dengan konfigurasi \textit{delay} 5 detik per \textit{header}, dibutuhkan waktu \textbf{$\sim$58 hari} untuk menyelesaikan pengiriman seluruh 1.000.000 \textit{garbage header} --- menghasilkan rasio \textit{time waste} mendekati tak terhingga.
    \item Setiap koneksi tarpit mengonsumsi \textit{socket file descriptor}, \textit{thread}, dan alokasi memori pada sisi penyerang, mengakibatkan \textit{connection pool saturation} secara progresif.
    \item \textit{Tool} otomatis (SQLMap, Nikto, DirBuster) tidak memiliki heuristik untuk mendeteksi bahwa respons yang diterima merupakan jebakan, karena \textit{status code} awal yang dikirimkan tetap valid (\texttt{HTTP 200 OK}).
\end{itemize}

\subsection{Integrasi REST API}

S.C.A.R. bertindak sebagai \textit{HTTP client} saat berkomunikasi dengan API eksternal:

\subsubsection{AbuseIPDB API}
\begin{lstlisting}[caption={Komunikasi dengan AbuseIPDB REST API}]
# HTTP GET Request ke AbuseIPDB
GET /api/v2/check?ipAddress=1.2.3.4 HTTP/1.1
Host: api.abuseipdb.com
Key: [API_KEY]
Accept: application/json

# HTTP Response dari AbuseIPDB
{
  "data": {
    "ipAddress": "1.2.3.4",
    "abuseConfidenceScore": 85,
    "totalReports": 42
  }
}
\end{lstlisting}

\subsubsection{Telegram Bot API}
\begin{lstlisting}[caption={Komunikasi dengan Telegram Bot API}]
# HTTP POST Request ke Telegram
POST /bot[TOKEN]/sendMessage HTTP/1.1
Host: api.telegram.org
Content-Type: application/json

{
  "chat_id": "[CHAT_ID]",
  "text": "THREAT ALERT: IP 1.2.3.4 ...",
  "parse_mode": "Markdown"
}
\end{lstlisting}

Kedua integrasi ini mendemonstrasikan bagaimana sebuah server jaringan dapat bertindak sebagai \textit{client} dan \textit{server} secara bersamaan --- menerima koneksi penyerang sambil mengirim data ke layanan luar.

\subsection{Threat Fusion Logic}

Hasil analisis dari 4 model AI digabungkan menggunakan logika fusi \textit{Hard/Soft Flag}:

\begin{table}[H]
\centering
\caption{Sistem Hard Flag dan Soft Flag}
\begin{tabular}{llcc}
\toprule
\textbf{Layer} & \textbf{Model AI} & \textbf{Tipe Flag} & \textbf{Bisa Trigger Tarpit Sendiri} \\
\midrule
Layer 2 & URL Threat & Hard & Ya \\
Layer 3 & SQL Injection & Hard & Ya \\
Layer 4 & HTTP Behavior & Soft & Tidak (butuh konfirmasi) \\
Layer 5 & Anomaly Detection & Soft & Tidak (butuh konfirmasi) \\
\bottomrule
\end{tabular}
\end{table}

\textbf{Aturan:}
\begin{itemize}
    \item $\geq 1$ Hard Flag $\rightarrow$ \textbf{Tarpit} langsung.
    \item $\geq 2$ Soft Flags $\rightarrow$ \textbf{Tarpit} (konfirmasi silang).
    \item 1 Soft Flag saja $\rightarrow$ \textbf{Warning} (dicatat, request dilayani).
\end{itemize}

Logika ini mencegah \textit{false positive} --- request normal yang kebetulan ditandai oleh satu model saja tidak akan memicu tarpit.

% ============================================================
% BAB 4: IMPLEMENTASI
% ============================================================
\clearpage
\section{Implementasi}

\subsection{Lingkungan Pengembangan}

\begin{table}[H]
\centering
\caption{Spesifikasi Lingkungan}
\begin{tabular}{ll}
\toprule
\textbf{Komponen} & \textbf{Detail} \\
\midrule
Bahasa & Python 3.8+ \\
Server Framework & \texttt{http.server} + \texttt{socketserver} (standar library) \\
HTTP Client & \texttt{requests} library \\
ML Libraries & scikit-learn, numpy, pandas \\
Concurrency & \texttt{threading} module \\
Version Control & Git + GitHub \\
\bottomrule
\end{tabular}
\end{table}

\subsection{Struktur Proyek}

\begin{lstlisting}[language={}, caption={Struktur Direktori Proyek}]
Project_SCAR/
|-- server.py                # HTTP Server + Fusion Engine + Tarpit
|-- attacker_simulation.py   # HTTP Client Simulasi Serangan
|-- default.html             # Honeypot HTML Page
|-- requirements.txt         # Dependencies
|-- models/                  # Pre-trained AI Models (.pkl)
|   |-- url_model.pkl
|   |-- url_vectorizer.pkl
|   |-- sqli_model.pkl
|   |-- sqli_vectorizer.pkl
|   |-- behavior_model.pkl
|   +-- anomaly_model.pkl
+-- README.md
\end{lstlisting}

\subsection{Tampilan Kode Server}

\begin{figure}[H]
\centering
\includegraphics[width=13cm]{ss_code.png}
\caption{Tampilan kode \texttt{server.py} di Visual Studio Code}
\label{fig:code}
\end{figure}

\subsection{Tampilan Halaman Honeypot}

Halaman HTML palsu disajikan kepada pengunjung normal melalui HTTP response standar.

\begin{figure}[H]
\centering
\includegraphics[width=13cm]{ss_honeypot.png}
\caption{Tampilan halaman honeypot di browser (\texttt{http://localhost:8000})}
\label{fig:honeypot}
\end{figure}

\subsection{Implementasi HTTP Server}

Server dibangun menggunakan \texttt{ThreadingHTTPServer} yang mewarisi \texttt{socketserver.ThreadingMixIn}, memungkinkan penanganan multi-koneksi:

\begin{lstlisting}[caption={Inisialisasi Server dengan Threading}]
class ThreadingHTTPServer(socketserver.ThreadingMixIn,
                          http.server.HTTPServer):
    daemon_threads = True

server = ThreadingHTTPServer(("0.0.0.0", 8000), CustomHandler)
server.serve_forever()
\end{lstlisting}

Kelas \texttt{CustomHandler} menangani setiap request dengan alur:
\begin{lstlisting}[caption={Handler utama -- Parsing HTTP Request}]
def handle_request(self):
    client_ip = self.client_address[0]   # IP dari socket
    method = self.command                  # GET, POST, dll
    path = self.path                       # URL path + query

    # Baca body dari socket (jika ada)
    content_length = int(self.headers.get('Content-Length', 0))
    body = self.rfile.read(content_length).decode('utf-8')

    # Proses melalui lapisan pertahanan...
\end{lstlisting}

\subsection{Implementasi Tarpit (Socket-Level)}

Ini adalah bagian \textit{core} dari perspektif pemrograman jaringan --- pengiriman data langsung ke \textit{socket} TCP:

\begin{lstlisting}[caption={Implementasi Tarpit melalui Socket Write}]
def execute_tarpit(self, client_ip, reasons, risk_score=0.0):
    # Kirim Telegram alert di background thread
    send_telegram_alert(client_ip, reasons, risk_score)

    # Tulis langsung ke socket melalui self.wfile
    self.wfile.write(b"HTTP/1.1 200 OK\r\n")
    self.wfile.write(b"Transfer-Encoding: chunked\r\n")
    for key, val in FAKE_HEADERS.items():
        self.wfile.write(f"{key}: {val}\r\n".encode())
    self.wfile.write(b"\r\n")

    # Kirim garbage headers secara kontinu
    count = 0
    while count < TARPIT_HEADER_COUNT:
        random_id = random.randint(100000, 999999)
        random_hex = hashlib.sha256(
            os.urandom(32)).hexdigest()
        garbage = f"X-Trap-{random_id}: {random_hex}\r\n"
        self.wfile.write(garbage.encode())
        self.wfile.flush()   # Force kirim ke socket
        count += 1
        time.sleep(TARPIT_DELAY_SECONDS)  # Jeda 5 detik
\end{lstlisting}

Perhatikan penggunaan \texttt{self.wfile.write()} dan \texttt{self.wfile.flush()} --- ini menulis byte langsung ke koneksi TCP socket yang terhubung ke klien.

\subsection{Implementasi Non-Blocking Telegram Alert}

Pengiriman alert menggunakan \textit{background thread} agar proses tarpit tidak terganggu:

\begin{lstlisting}[caption={Non-Blocking API Call dengan Threading}]
def send_telegram_alert(ip, reasons, risk_score=0.0):
    def _send():
        url = f"https://api.telegram.org/bot{TOKEN}/sendMessage"
        data = {"chat_id": CHAT_ID,
                "text": format_message(ip, reasons, risk_score),
                "parse_mode": "Markdown"}
        requests.post(url, json=data, timeout=5)

    # Jalankan di background thread
    threading.Thread(target=_send, daemon=True).start()
\end{lstlisting}

Ini mendemonstrasikan pola \textit{Fire-and-Forget} dalam pemrograman jaringan --- mengirim request tanpa menunggu hasilnya, agar \textit{main thread} tetap fokus pada koneksi tarpit.

\subsection{Implementasi IP Reputation Check}

\begin{lstlisting}[caption={REST API Call ke AbuseIPDB dengan Caching}]
IP_CACHE = {}  # Dictionary untuk caching
IP_CACHE_LOCK = threading.Lock()  # Thread-safe access

def check_ip_reputation(ip):
    with IP_CACHE_LOCK:
        if ip in IP_CACHE:
            cached = IP_CACHE[ip]
            if time.time() - cached['time'] < CACHE_TTL:
                return cached['result']

    # HTTP GET request ke AbuseIPDB
    headers = {"Key": ABUSEIPDB_API_KEY,
               "Accept": "application/json"}
    resp = requests.get(
        f"https://api.abuseipdb.com/api/v2/check",
        params={"ipAddress": ip, "maxAgeInDays": 90},
        headers=headers, timeout=5)
    data = resp.json()["data"]

    # Cache hasil dengan thread-safe lock
    with IP_CACHE_LOCK:
        IP_CACHE[ip] = {"result": data, "time": time.time()}
    return data
\end{lstlisting}

Caching dengan \texttt{threading.Lock()} mendemonstrasikan penanganan \textit{race condition} --- ketika beberapa \textit{thread} mencoba mengakses cache secara bersamaan.

% ============================================================
% BAB 5: PENGUJIAN DAN HASIL
% ============================================================
\clearpage
\section{Pengujian dan Hasil}

\subsection{Skenario Pengujian}

Pengujian dilakukan menggunakan \texttt{attacker\_simulation.py}, sebuah HTTP client yang mengirim berbagai jenis request ke server S.C.A.R.:

\begin{table}[H]
\centering
\caption{Skenario Pengujian}
\begin{tabularx}{\textwidth}{clXc}
\toprule
\textbf{No} & \textbf{Skenario} & \textbf{HTTP Request} & \textbf{Ekspektasi} \\
\midrule
1 & Normal User & \texttt{GET / HTTP/1.1} & HTTP 200 + HTML \\
2 & Normal User & \texttt{GET /images/logo.png HTTP/1.1} & HTTP 200 + HTML \\
3 & Recon Attack & \texttt{GET /cgi-bin/...?val=../../../../bin/ls} & Tarpit \\
4 & SQL Injection & \texttt{GET /?cat=1 AND 1=1 HTTP/1.1} & Tarpit \\
5 & Anomaly & \texttt{GET /search?q=><;()@!\#... HTTP/1.1} & Tarpit \\
\bottomrule
\end{tabularx}
\end{table}

\subsection{Hasil Pengujian}

\begin{table}[H]
\centering
\caption{Hasil Pengujian}
\begin{tabularx}{\textwidth}{clccl}
\toprule
\textbf{No} & \textbf{Skenario} & \textbf{Hasil} & \textbf{Risk} & \textbf{Respons Jaringan} \\
\midrule
1 & Normal GET / & \textcolor{green}{\textbf{Lolos}} & 0.44 & HTTP 200 + HTML body \\
2 & Normal /login & \textcolor{green}{\textbf{Lolos}} & 0.25 & HTTP 200 + HTML body \\
3 & Recon (/etc/passwd) & \textcolor{red}{\textbf{Tarpit}} & >0.80 & HTTP 200 + garbage stream \\
4 & SQL Injection & \textcolor{red}{\textbf{Tarpit}} & 0.56 & HTTP 200 + garbage stream \\
5 & Anomalous & \textcolor{red}{\textbf{Tarpit}} & 0.52 & HTTP 200 + garbage stream \\
\bottomrule
\end{tabularx}
\end{table}

\subsection{Bukti Pengujian}

\subsubsection{Terminal Server}
Berikut adalah tampilan terminal server saat menerima dan memproses request serangan:

\begin{figure}[H]
\centering
\includegraphics[width=\textwidth]{ss_server.png}
\caption{Output terminal server saat mendeteksi ancaman dan mengaktifkan tarpit}
\label{fig:server}
\end{figure}

\subsubsection{Terminal Attacker Simulation}
Berikut adalah output dari sisi penyerang (\texttt{attacker\_simulation.py}):

\begin{figure}[H]
\centering
\includegraphics[width=13cm]{ss_attacker.png}
\caption{Output simulasi serangan --- request normal berhasil, request serangan terjebak tarpit}
\label{fig:attacker}
\end{figure}

\subsubsection{Notifikasi Telegram}
Setiap ancaman yang terdeteksi memicu pengiriman alert ke grup Telegram melalui REST API:

\begin{figure}[H]
\centering
\includegraphics[width=13cm]{ss_telegram.png}
\caption{Alert S.C.A.R. yang diterima di grup Telegram}
\label{fig:telegram}
\end{figure}

\subsection{Analisis Komunikasi Jaringan}

\subsubsection{Request Normal}
Komunikasi jaringan untuk request normal berjalan standar:
\begin{lstlisting}[language={}, caption={Pertukaran Data Jaringan - Request Normal}]
CLIENT >> GET / HTTP/1.1
CLIENT >> Host: localhost:8000
CLIENT >> User-Agent: python-requests/2.31

SERVER << HTTP/1.1 200 OK
SERVER << Content-Type: text/html
SERVER << Content-Length: 2272
SERVER << [HTML body - halaman honeypot]
\end{lstlisting}
Waktu respons: $\sim$2 detik (termasuk analisis AI).

\subsubsection{Request Serangan --- Tarpit Aktif}
Komunikasi jaringan untuk request serangan menunjukkan mekanisme tarpit:
\begin{lstlisting}[language={}, caption={Pertukaran Data Jaringan - Tarpit Aktif}]
CLIENT >> GET /?cat=1 AND 1=1 HTTP/1.1
CLIENT >> Host: localhost:8000

SERVER << HTTP/1.1 200 OK
SERVER << Transfer-Encoding: chunked
SERVER << X-Trap-482917: a3f8c1d2e5b7...\r\n  [+0s]
SERVER << X-Trap-193847: 7bc2e5f9a1d3...\r\n  [+5s]
SERVER << X-Trap-572910: 1de9f3a8c4b6...\r\n  [+10s]
   ... (berlanjut tanpa batas)

CLIENT >> ConnectionError: InvalidChunkLength
   (klien gagal mem-parse garbage sebagai chunk yang valid)
\end{lstlisting}

Error \texttt{InvalidChunkLength} pada sisi klien mengkonfirmasi bahwa data tarpit berhasil dikirim melalui \textit{socket} dan klien tidak dapat memproses respons --- koneksi terjebak.

\subsubsection{Notifikasi Telegram}
Setiap tarpit memicu HTTP POST ke Telegram API di \textit{background thread}. Rata-rata latensi pengiriman: $<$1 detik. Pengiriman non-blocking memastikan tarpit tidak tertunda.

\subsection{Analisis Efektivitas}

Berdasarkan hasil pengujian yang telah dilaksanakan, berikut merupakan analisis kuantitatif dan kualitatif terhadap efektivitas sistem S.C.A.R.:

\begin{enumerate}[label=\arabic*.]
    \item \textbf{\textit{Zero False Positive Rate}}: Seluruh \textit{request} normal (skenario 1 dan 2) berhasil dilayani dengan respons \texttt{HTTP 200 OK} beserta konten HTML yang utuh. Hal ini memvalidasi bahwa logika fusi \textit{Hard/Soft Flag} pada \textit{Threat Fusion Engine} mampu membedakan lalu lintas legitimate dari lalu lintas berbahaya secara akurat.

    \item \textbf{\textit{Detection Rate} 100\%}: Ketiga kategori serangan (\textit{reconnaissance}, \textit{SQL Injection}, dan \textit{anomalous request}) berhasil diidentifikasi dan dialihkan ke mekanisme tarpit. Skor risiko bervariasi dari 0,52 (anomali) hingga 0,90 (\textit{reconnaissance}), mengindikasikan bahwa sistem memberikan gradasi ancaman yang proporsional terhadap tingkat keparahan serangan.

    \item \textbf{Efektivitas Tarpit}: Pengujian menunjukkan bahwa mekanisme \textit{chunked transfer encoding} berhasil menahan koneksi penyerang selama 10--30 detik sebelum terjadi \textit{timeout} dengan \textit{error} \texttt{InvalidChunkLength}. Melalui mekanisme ini, latensi penyerang meningkat secara signifikan dibandingkan respons normal ($\sim$2 detik), sehingga efektivitas \textit{scanning} terdegradasi hingga mendekati 99\%.

    \item \textbf{Stabilitas \textit{Multi-Threading}}: Server tetap responsif dalam melayani \textit{request} normal meskipun terdapat koneksi yang sedang dalam proses tarpit secara bersamaan. Arsitektur \texttt{ThreadingTCPServer} dengan \texttt{daemon\_threads = True} memastikan bahwa setiap koneksi dikelola secara independen tanpa terjadi \textit{thread starvation}.

    \item \textbf{Reliabilitas Integrasi API}: Komunikasi dengan AbuseIPDB (\textit{threat intelligence}) dan Telegram Bot API (notifikasi \textit{real-time}) berfungsi dengan latensi rata-rata $<$1 detik. Pola \textit{fire-and-forget} pada \textit{background thread} memastikan bahwa operasi I/O eksternal tidak menimbulkan \textit{bottleneck} pada proses tarpit.
\end{enumerate}

% ============================================================
% BAB 6: KESIMPULAN DAN SARAN
% ============================================================
\clearpage
\section{Kesimpulan dan Saran}

\subsection{Kesimpulan}

Berdasarkan hasil perancangan, implementasi, dan pengujian yang telah dilaksanakan, dapat ditarik kesimpulan sebagai berikut:

\begin{enumerate}[label=\arabic*.]
    \item Proyek S.C.A.R. berhasil mengimplementasikan konsep-konsep inti \textbf{pemrograman jaringan} secara terintegrasi, meliputi: \textit{HTTP server}, \textit{socket programming}, siklus \textit{request-response}, \textit{chunked transfer encoding}, integrasi REST API, dan \textit{multi-threading} dalam satu sistem pertahanan aktif yang kohesif.

    \item Mekanisme \textbf{Tarpit} melalui teknik \textit{HTTP response stream manipulation} membuktikan bahwa eksploitasi karakteristik protokol HTTP/1.1 (khususnya \textit{chunked encoding}) dapat digunakan secara defensif untuk menguras sumber daya penyerang. Pengujian menunjukkan bahwa mekanisme ini mampu meningkatkan latensi penyerang secara signifikan, menyebabkan \textit{tool} pemindaian otomatis mengalami \textit{timeout}.

    \item Arsitektur \textit{dual-role} server --- bertindak sebagai \textbf{\textit{server}} (menerima koneksi penyerang) dan \textbf{\textit{client}} (berkomunikasi dengan AbuseIPDB dan Telegram API) secara simultan --- mendemonstrasikan konsep \textit{bidirectional network communication} dalam aplikasi nyata.

    \item Penerapan \textbf{\textit{multi-threading}} melalui \texttt{ThreadingTCPServer} memungkinkan penanganan koneksi tarpit berdurasi panjang tanpa mengganggu responsivitas server terhadap koneksi lainnya, memvalidasi implementasi \textit{concurrency} yang tepat dalam arsitektur server jaringan.

    \item Integrasi \textbf{\textit{Machine Learning}} melalui arsitektur \textit{Threat Fusion Engine} dengan mekanisme \textit{Hard/Soft Flag} memberikan kemampuan deteksi otomatis yang adaptif, melampaui keterbatasan sistem berbasis aturan statis. Pengujian mencatat \textit{zero false positive} pada lalu lintas normal, mengonfirmasi reliabilitas logika fusi.
\end{enumerate}

Berdasarkan keseluruhan hasil pengujian, S.C.A.R. terbukti mampu mengubah paradigma pertahanan dari pendekatan pasif (sekadar mencatat \textit{log} serangan) menjadi pertahanan aktif (merespons balik penyerang secara \textit{real-time}). Implementasi \textit{Multi-Layer AI} berhasil menekan tingkat \textit{false positive} hingga nol, sementara mekanisme tarpit secara efektif menghabiskan waktu dan sumber daya penyerang, memberikan waktu yang berharga bagi administrator jaringan untuk melakukan langkah mitigasi lebih lanjut.

\subsection{Saran Pengembangan}

Beberapa rekomendasi pengembangan untuk meningkatkan kapabilitas sistem di masa mendatang:

\begin{enumerate}[label=\arabic*.]
    \item \textbf{\textit{Asynchronous I/O}}: Migrasi ke arsitektur \texttt{asyncio} untuk meningkatkan efisiensi penanganan koneksi pada volume tinggi, menggantikan model \textit{threading} yang memiliki \textit{overhead} pembuatan \textit{thread} per koneksi.
    \item \textbf{\textit{SSL/TLS Encryption}}: Penambahan dukungan HTTPS menggunakan modul \texttt{ssl} untuk mengamankan komunikasi serta meningkatkan autentisitas \textit{deception} terhadap penyerang.
    \item \textbf{\textit{WebSocket Real-Time Dashboard}}: Pembangunan antarmuka pemantauan \textit{real-time} berbasis protokol WebSocket untuk visualisasi aktivitas serangan dan status tarpit secara langsung.
    \item \textbf{\textit{Rate Limiting}}: Implementasi mekanisme pembatasan jumlah koneksi per IP menggunakan algoritma \textit{token bucket} untuk lapisan pertahanan tambahan.
    \item \textbf{Kontainerisasi}: \textit{Deployment} menggunakan Docker untuk mempermudah replikasi dan skalabilitas di lingkungan produksi.
\end{enumerate}

% ============================================================
% DAFTAR PUSTAKA
% ============================================================
\newpage
\section*{Daftar Pustaka}
\addcontentsline{toc}{section}{Daftar Pustaka}

\begin{enumerate}[label={[\arabic*]}]

\item Fielding, R., et al. (1999). \textit{Hypertext Transfer Protocol -- HTTP/1.1}. RFC 2616, IETF. --- Spesifikasi protokol HTTP, termasuk \textit{chunked transfer encoding} yang digunakan pada mekanisme tarpit.

\item Spitzner, L. (2003). \textit{Honeypots: Tracking Hackers}. Addison-Wesley Professional. --- Referensi utama tentang konsep dan arsitektur honeypot.

\item Liston, T. (2003). \textit{LaBrea: ``Sticky'' Honeypot and IDS}. --- Implementasi awal konsep tarpit untuk melawan worm jaringan, dasar mekanisme \textit{resource exhaustion} pada S.C.A.R.

\item Provos, N., \& Holz, T. (2007). \textit{Virtual Honeypots: From Botnet Tracking to Intrusion Detection}. Addison-Wesley. --- Arsitektur honeypot virtual dan teknik deteksi intrusi.

\item Stevens, W. R. (1994). \textit{TCP/IP Illustrated, Volume 1: The Protocols}. Addison-Wesley. --- Referensi fundamental tentang protokol TCP/IP dan socket programming.

\item Beazley, D. M. (2009). \textit{Python Essential Reference}. Addison-Wesley. --- Referensi modul \texttt{http.server}, \texttt{socketserver}, dan \texttt{threading} Python.

\item Pedregosa, F., et al. (2011). Scikit-learn: Machine Learning in Python. \textit{Journal of Machine Learning Research}, 12, 2825--2830. --- Pustaka ML yang digunakan untuk implementasi model AI.

\item Breiman, L. (2001). Random Forests. \textit{Machine Learning}, 45(1), 5--32. --- Algoritma Random Forest untuk model HTTP Behavior.

\item Liu, F. T., Ting, K. M., \& Zhou, Z. H. (2008). Isolation Forest. \textit{Proc. IEEE ICDM}, 413--422. --- Algoritma Isolation Forest untuk deteksi anomali Zero-Day.

\item Buczak, A. L., \& Guven, E. (2016). A Survey of Data Mining and Machine Learning Methods for Cyber Security Intrusion Detection. \textit{IEEE Comm. Surveys \& Tutorials}, 18(2), 1153--1176. --- Survei teknik ML untuk keamanan jaringan.

\item OWASP Foundation. (2024). \textit{OWASP Testing Guide v4}. \url{https://owasp.org} --- Panduan pengujian keamanan, referensi pola \textit{reconnaissance}.

\item AbuseIPDB. (2024). \textit{API Documentation}. \url{https://www.abuseipdb.com/api.html} --- Dokumentasi REST API untuk \textit{threat intelligence}.

\item Telegram. (2024). \textit{Bot API}. \url{https://core.telegram.org/bots/api} --- Dokumentasi REST API untuk notifikasi \textit{real-time}.

\item Verizon. (2024). \textit{Data Breach Investigations Report}. --- Statistik tren serangan siber.

\item Python Software Foundation. (2024). \textit{http.server --- HTTP servers}. \url{https://docs.python.org/3/library/http.server.html} --- Dokumentasi modul \texttt{http.server} Python.

\end{enumerate}

\end{document}


% ============================================================
% DAFTAR PUSTAKA
% ============================================================
\newpage
\begin{thebibliography}{99}
\addcontentsline{toc}{section}{Daftar Pustaka}

\bibitem{spitzner2003} 
Spitzner, L. (2003). \textit{Honeypots: Tracking Hackers}. Addison-Wesley Professional.

\bibitem{liston2003} 
Liston, T. (2003). \textit{LaBrea Tarpit: Sticky Honeypots}. Available at: \url{http://labrea.sourceforge.net/}.

\bibitem{breiman2001} 
Breiman, L. (2001). Random Forests. \textit{Machine Learning}, 45(1), 5-32.

\bibitem{liu2008} 
Liu, F. T., Ting, K. M., \& Zhou, Z. H. (2008). Isolation Forest. \textit{Eighth IEEE International Conference on Data Mining}, 413-422.

\bibitem{verizon2024} 
Verizon. (2024). \textit{2024 Data Breach Investigations Report (DBIR)}. Verizon Business.

\bibitem{rfc2616} 
Fielding, R., et al. (1999). \textit{Hypertext Transfer Protocol -- HTTP/1.1}. RFC 2616, IETF.

\bibitem{python3} 
Python Software Foundation. (2024). \textit{Python Standard Library Documentation: http.server \& socketserver}. Available at: \url{https://docs.python.org/3/library/}.

\end{thebibliography}

\end{document}
